%% Generated by Sphinx.
\def\sphinxdocclass{ujreport}
\documentclass[a4paper,12pt,dvipdfmx]{sphinxhowto}
\ifdefined\pdfpxdimen
   \let\sphinxpxdimen\pdfpxdimen\else\newdimen\sphinxpxdimen
\fi \sphinxpxdimen=.75bp\relax
\ifdefined\pdfimageresolution
    \pdfimageresolution= \numexpr \dimexpr1in\relax/\sphinxpxdimen\relax
\fi
%% let collapsible pdf bookmarks panel have high depth per default
\PassOptionsToPackage{bookmarksdepth=5}{hyperref}

\PassOptionsToPackage{warn}{textcomp}


\usepackage{cmap}
\usepackage[T1]{fontenc}
\usepackage{amsmath,amssymb,amstext}




\usepackage{tgtermes}
\usepackage{tgheros}
\renewcommand{\ttdefault}{txtt}




\usepackage{sphinx}

\fvset{fontsize=auto}
\usepackage[dvipdfm]{geometry}


% Include hyperref last.
\usepackage{hyperref}
% Fix anchor placement for figures with captions.
\usepackage{hypcap}% it must be loaded after hyperref.
% Set up styles of URL: it should be placed after hyperref.
\urlstyle{same}
\usepackage{pxjahyper}

\renewcommand{\contentsname}{Contents:}

\usepackage{sphinxmessages}
\setcounter{tocdepth}{2}



\title{PythonCA:Python module for EPICS CA library Documentation}
\date{2022年05月24日}
\release{1.22}
\author{Noboru Yamamoto}
\newcommand{\sphinxlogo}{\vbox{}}
\renewcommand{\releasename}{リリース}
\makeindex
\begin{document}

\pagestyle{empty}
\sphinxmaketitle
\pagestyle{plain}
\sphinxtableofcontents
\pagestyle{normal}
\phantomsection\label{\detokenize{index::doc}}


\sphinxstepscope


\section{Python\sphinxhyphen{}CA  installation memo}
\label{\detokenize{InstallationMemo:python-ca-installation-memo}}\label{\detokenize{InstallationMemo::doc}}
\sphinxAtStartPar
KEK, High Energy Accelerator Research Organization
Acceleator Lab
Noboru Yamamoto

\begin{DUlineblock}{0em}
\item[] 
\end{DUlineblock}
\begin{enumerate}
\sphinxsetlistlabels{\arabic}{enumi}{enumii}{}{.}%
\item {} 
\sphinxAtStartPar
This program is retistered in PyPI.
\begin{enumerate}
\sphinxsetlistlabels{\arabic}{enumii}{enumiii}{}{.}%
\item {} 
\sphinxAtStartPar
you need to setup EPICSROOT environment varible before install this program
using pip command. You may need other environment variables, such as WITH\_TK, TKINC, TKLIB, HOSTARCH.

\item {} 
\sphinxAtStartPar
You can create create EPICS\_config\_local.py files to setup these environment. Put this file in
somewhere in your PYTHONPATH.

\end{enumerate}

\item {} 
\sphinxAtStartPar
You need Python 2.7 or later and EPICS 3.14.7 or later. (It may work
with older version, but these are oldest versions I have built.)

\item {} 
\sphinxAtStartPar
Get a Python\sphinxhyphen{}CA extension module package as \sphinxhref{CaPython-1.10.tar.gz}{a tarball
here}{[}updated on 2007/05/03{]}.

\item {} 
\sphinxAtStartPar
 Expand this tarball at  your working directory.

\item {} 
\sphinxAtStartPar
Open setup.py in your favourite editor and change some parametes,
such as EPICS architechture and installation path, appropriately.

\begin{sphinxVerbatim}[commandchars=\\\{\}]
\PYG{n}{EPICSROOT}\PYG{o}{=}\PYG{n}{os}\PYG{o}{.}\PYG{n}{path}\PYG{o}{.}\PYG{n}{join}\PYG{p}{(}\PYG{l+s+s2}{\PYGZdq{}}\PYG{l+s+s2}{your epics root path}\PYG{l+s+s2}{\PYGZdq{}}\PYG{p}{)}
\end{sphinxVerbatim}

\begin{sphinxVerbatim}[commandchars=\\\{\}]
\PYG{n}{EPICSBASE}\PYG{o}{=}\PYG{n}{os}\PYG{o}{.}\PYG{n}{path}\PYG{o}{.}\PYG{n}{join}\PYG{p}{(}\PYG{n}{EPICSROOT}\PYG{p}{,}\PYG{l+s+s2}{\PYGZdq{}}\PYG{l+s+s2}{base}\PYG{l+s+s2}{\PYGZdq{}}\PYG{p}{)}
\end{sphinxVerbatim}

\begin{sphinxVerbatim}[commandchars=\\\{\}]
\PYG{n}{EPICSEXT}\PYG{o}{=}\PYG{n}{os}\PYG{o}{.}\PYG{n}{path}\PYG{o}{.}\PYG{n}{join}\PYG{p}{(}\PYG{n}{EPICSROOT}\PYG{p}{,}\PYG{l+s+s2}{\PYGZdq{}}\PYG{l+s+s2}{extensions}\PYG{l+s+s2}{\PYGZdq{}}\PYG{p}{)}
\end{sphinxVerbatim}

\begin{sphinxVerbatim}[commandchars=\\\{\}]
\PYG{n}{HOSTARCH}\PYG{o}{=}\PYG{l+s+s2}{\PYGZdq{}}\PYG{l+s+s2}{your epics host architecture}\PYG{l+s+s2}{\PYGZdq{}}
\end{sphinxVerbatim}

\item {} 
\sphinxAtStartPar
run the installation script, setup.py/  for build extension moules.

\begin{sphinxVerbatim}[commandchars=\\\{\}]
\PYG{n}{python} \PYG{n}{setup}\PYG{o}{.}\PYG{n}{py} \PYG{n}{build}
\end{sphinxVerbatim}

\item {} 
\sphinxAtStartPar
if you encounter the compilation errors or any trouble , please send
a message to  noboru.yamamoto\_at\_kek.jp.

\item {} 
\begin{DUlineblock}{0em}
\item[] You need to have write permission of the target directories for
installation. Run:
\end{DUlineblock}

\begin{sphinxVerbatim}[commandchars=\\\{\}]
\PYG{n}{python} \PYG{n}{setup}\PYG{o}{.}\PYG{n}{py} \PYG{n}{install}
\end{sphinxVerbatim}

\item {} 
\sphinxAtStartPar
Test extension module.

\end{enumerate}

\sphinxAtStartPar
start python interpreter.

\begin{sphinxVerbatim}[commandchars=\\\{\}]
\PYG{n}{python}
\end{sphinxVerbatim}

\sphinxAtStartPar
Try to import ca module

\begin{sphinxVerbatim}[commandchars=\\\{\}]
\PYG{k+kn}{import} \PYG{n+nn}{ca}
\end{sphinxVerbatim}

\sphinxAtStartPar
Check access to EPICS DB. (Assuming excas is running.)

\begin{sphinxVerbatim}[commandchars=\\\{\}]
\PYG{n}{ca}\PYG{o}{.}\PYG{n}{Get}\PYG{p}{(}\PYG{l+s+s2}{\PYGZdq{}}\PYG{l+s+s2}{fred}\PYG{l+s+s2}{\PYGZdq{}}\PYG{p}{)}
\end{sphinxVerbatim}


\subsection{Note to a GUI programmer:}
\label{\detokenize{InstallationMemo:note-to-a-gui-programmer}}
\begin{DUlineblock}{0em}
\item[] You should not call functions in GUI system(Tkinter or wxPython) in
the Python\sphinxhyphen{}CA callback routines. It will crash your running program
immediately.
\end{DUlineblock}

\sphinxstepscope


\section{Relase Note for version 1.23.2.2.1}
\label{\detokenize{ReleaseMemo-1.23.2.2.1:relase-note-for-version-1-23-2-2-1}}\label{\detokenize{ReleaseMemo-1.23.2.2.1::doc}}
\sphinxAtStartPar
これ以前のPython CAでWaveform レコードにputを繰り返すと、当初は正常に
動作しているが、2000\sphinxhyphen{}3000回程度を超えたところで、
\begin{quote}

\sphinxAtStartPar
Fatal Python error: deallocating None
\end{quote}

\sphinxAtStartPar
とのエラーメッセージを残して、プロセスがクラッシュするという現象が発生することが判明した。

\sphinxAtStartPar
これに対応するために、\sphinxcode{\sphinxupquote{Py\_XDECREF()}} 呼び出しの際に参照回数を減らす対象のオブジェクト
が:py:class:\sphinxtitleref{Py\_None} で無いことをチェックするように \_ca314.cpp を変更した。

\sphinxAtStartPar
\sphinxcode{\sphinxupquote{Py\_None}} に対しても:py:func:\sphinxtitleref{Py\_INCREF} を呼んでおり、\sphinxcode{\sphinxupquote{Py\_None}} に対して:py:func:\sphinxtitleref{Py\_DECREF} を呼んだ場合、
いつでもクラッシュする訳ではないようなので、完全には理解できていないが、クラッシュすることは
なくなった。 \sphinxcode{\sphinxupquote{Py\_None}} に対して \sphinxcode{\sphinxupquote{Py\_DECREF()}} 呼び出しをかけようとした時にメッセージを出す様にしてもこのメッセージは表示されない。


\section{Indices and tables}
\label{\detokenize{index:indices-and-tables}}\begin{itemize}
\item {} 
\sphinxAtStartPar
\DUrole{xref,std,std-ref}{genindex}

\item {} 
\sphinxAtStartPar
\DUrole{xref,std,std-ref}{modindex}

\item {} 
\sphinxAtStartPar
\DUrole{xref,std,std-ref}{search}

\end{itemize}



\renewcommand{\indexname}{索引}
\printindex
\end{document}